\section{Laboratory Methods}
This section is the most vague of all the PGRE topics.
I put information down that helped me, but please contact me if you notice a glaring omission or know more about this than I do (very likely).

\subsection{Dimensional Analysis}
Know how to deduce if a solution has the correct units.\\*
Also understand if a solution is reasonable (i.e. make sure the velocity you calculated is less than or equal to the speed of light).

\subsection{Poisson Distribution}
Also called the law of small numbers.\\\\*
\(\displaystyle p(k)=\frac{\lambda^k}{k!e^{\lambda}}\)\\*
\(\lambda\) is the rate at which the (rare) event occurs\\\\*
The mean and variance of the distribution are the same.
\begin{itemize}
\item The Poisson distribution describes mutually independent events, occurring at a known and constant rate (\(\lambda\)) per unit (time or space), and observed through a certain window: a unit of time or space
\item The probability of \(k\) occurrences in that unit can be calculated from \(p(k)\)
\item The rate is also the expected or most likely outcome (for whole number \(\lambda\) greater than 1, the outcome corresponding to \(\lambda-1\) is equally likely)
\end{itemize}
(This information was taken from the University of Massachusetts Amherst website on statistics.)

\subsection{Oscilloscopes}
Know how to read and interpret output from an oscilloscope including Lissajous curves.
