\section{Statistical Mechanics}
This is a brief summary of statistical mechanics for the PGRE and covers the bare essentials of this topic.
Most of the subtopics here are frequently tested, but the section over solids is mostly to get a qualitative understanding of energy levels and so on.
The most important topic, I feel, is the canonical ensemble and its partition function.
Again, it's a good idea to flesh out this section with your own graphs and diagrams.

\subsection{Theorem of Equipartition of Energy}
Each degree of freedom which contributes a quadratic term to the total energy has an average energy \(\frac{1}{2}k_BT\) and contributes \(\frac{k_B}{2}\) to the heat capacity (at constant volume, \(C_V\))\\\\*
For diatomic molecule:\\*
3 translational degrees of freedom: \(H_{trans}=\frac{1}{2}m\left(\dot{x}^2+\dot{y}^2+\dot{z}^2\right)\)\\*
2 vibrational degrees of freedom: \(H_{vib}=\frac{1}{2}m\left(\dot{x}^2+\omega^2x^2\right)\)\\*
2 rotational degrees of freedom: \(H_{rot}=\frac{1}{2}I\left(\omega_1^2+\omega_2^2\right)\) (other rotational axis has extremely small moment of inertia)\\*
Total degrees of freedom: 7

\subsection{Gases}

\subsubsection{Maxwell-Boltzmann Speed Distribution Function}
\(\displaystyle N_v=4\pi N\left(\frac{mv^2}{2\pi k_BT}\right)^{3/2}e^{-mv^2/2k_BT}\)

\subsubsection{Velocities}
Root-Mean Square velocity:\\*
\(\displaystyle K_{trans}=\frac{1}{2}mv_{rms}^2=\frac{3}{2}k_BT\rightarrow v_{rms}=\sqrt{\frac{3k_BT}{m}}\)\\\\*
Average Speed:\\*
\(\displaystyle \bar{v}=\sqrt{\frac{8k_BT}{\pi m}}\)\\\\*
Most Probable Speed:\\*
\(\displaystyle v_{mp}=\sqrt{\frac{2k_BT}{m}}\)\\\\*
Relationship Between Speeds:\\*
\(v_{rms}>\bar{v}>v_{mp}\)

\subsection{Radiation}
\subsubsection{Stefan's Law}
\(P=\sigma A e T^4\), where \(P\) is power,  \(A\) is the surface area, and \(e\) is the emissivity (the fraction of incoming radiation that the surface absorbs)\\*
If the surroundings are at \(T_0\) then \(P=\sigma A e (T^4-T_0^4)\)

\subsubsection{Blackbody Radiation}
\(e=1 \rightarrow P=\sigma A T^4\)\\*
Wien's displacement law: \(\lambda_{max} T\approx.003(mK)\), where \(\lambda_{max}\) is the maximum wavelength of light emitted from a blackbody at temperature \(T\)

\subsubsection{Ideal Reflector}
\(e=0\rightarrow P=0\)

\subsection{The Canonical \& Grand Canonical Ensembles}

\subsubsection{The Canonical Ensemble}
Partition function: \(\displaystyle Z=\sum_i{g_ie^{-E_i/k_BT}}\), where \(g_i\) is the degeneracy of state \(i\)\\*
Probability of system to be in state \(i\): \(\displaystyle p_i=\frac{g_ie^{-E_i/k_BT}}{Z}\)\\*
Ratio of probabilities: \(\displaystyle \frac{p_i}{p_j}=\frac{g_i}{g_j}e^{-(E_i-E_j)/k_BT}\)\\*
Entropy: \(\displaystyle S=-k_B\sum_i{p_i\ln{p_i}}\)\\*
Helmholtz free energy: \(F=-k_BT\ln{Z}\)\\*
Entropy from free energy: \(\displaystyle S=-\left(\frac{\partial F}{\partial T}\right)_V=k_B\ln{Z}+k_BT\frac{\partial\ln{Z}}{\partial T}\)\\*
Average internal energy: \(\displaystyle \bar{U}=\sum_i{p_iE_i}=k_BT^2\left(\frac{\partial\ln{Z}}{\partial T}\right)_V=k_B\frac{T^2}{Z}\left(\frac{\partial Z}{\partial T}\right)_V\)

\subsubsection{The Grand Canonical Ensemble}
Grand partition function: \(\displaystyle \Xi=\sum_i{g_ie^{-(E_i-\mu N_i)/k_BT}}\)\\*
Probability of system to be in state \(i\): \(\displaystyle p_i=\frac{g_ie^{-(\epsilon_i-\mu)/k_BT}}{\Xi}\)\\*
Grand potential: \(\Phi_G=-k_BT\ln{\Xi}=\bar{U}-\mu\bar{N}-TS\)\\*
Thermodynamic quantities:\\*
\(\displaystyle S=-\left(\frac{\partial\Phi_G}{\partial T}\right)_{V,\mu}\), \(\displaystyle P=-\left(\frac{\partial\Phi_G}{\partial V}\right)_{T,\mu}\), and \(\displaystyle \bar{N}=-\left(\frac{\partial\Phi_G}{\partial \mu}\right)_{V,T}\)

\subsection{Number Density}
\(\displaystyle n_V(E)=n_0e^{\frac{-E}{k_BT}}\)

\subsection{Symmetry and Statistics}
If a system starts in a symmetric/anti-symmetric state it must stay in a symmetric/anti-symmetric state.

\subsubsection{Symmetric State}
\(\psi(x_1,x_2)=\psi(x_2,x_1)\)\\\\*
Bosons are symmetric (photons, mesons, \(^4\)He).\\*
e.g.: \(\psi_{Bose}(x_1,x_2)=\phi_i(x_1)\phi_j(x_2)+\phi_i(x_2)\phi_j(x_1)\)\\\\*
%
Bose-Einstein distribution function:\\*
Distribution function: \(\displaystyle f(k)=\frac{1}{e^{(\epsilon(k)-\mu)/k_BT}-1}\), where \(\epsilon(k)-\mu>0\)

\subsubsection{Anti-Symmetric State}
\(\psi(x_1,x_2)=-\psi(x_2,x_1)\)\\\\*
Fermions are anti-symmetric (electrons, neutrinos, protons, \(^3\)He).\\*
e.g.: \(\psi_{Fermi}(x_1,x_2)=\phi_i(x_1)\phi_j(x_2)-\phi_i(x_2)\phi_j(x_1)\)\\\\*
%
Fermi-Dirac distribution function:\\*
\(\displaystyle\epsilon(k)=\frac{\hbar^2k^2}{2m}\)\\*
Distribution function: \(\displaystyle n(k)=\frac{1}{e^{(\epsilon(k)-\mu)/k_BT}+1}\)\\*
At high temperatures this reverts back to a Boltzmann distribution: \(\displaystyle n(k)\rightarrow e^{-(\epsilon(k)-\mu)/k_BT}\)\\*
At low temperatures \(n(k)\rightarrow1\)
%
\subsubsection{Fermi Gas}
\(n=\frac{\bar{N}}{V}\)\\*
Fermi wave number: \(k_F=\left(3\pi^2n\right)^{1/3}\)\\*
Fermi energy: \(\displaystyle E_F=\frac{\hbar^2k_F^2}{2m}\)\\*
Fermi temperature: \(\displaystyle T_F=\frac{E_F}{k_B}\)\\*
Fermi velocity: \(\displaystyle v_F=\frac{\hbar k_F}{m}\)\\\\*
%
In the high temperature limit: \(P=nk_BT\)\\\\*
When \(T>T_F\), \(\displaystyle n(k)\rightarrow e^{-(\epsilon(k)-\mu)/k_BT}\)\\*
When \(T\ll T_F\), one can assume the system is in its ground state; all electrons have energies less than or equal to the Fermi energy\\*
At \(T=0\): \(\displaystyle P=\frac{2nE_F}{5}\) and \(\displaystyle \bar{U}=\frac{3E_F}{5}\)\\*
At low \(T\): \(\displaystyle C_V=\frac{Nk_BT^2}{2}\left(\frac{k_BT}{E_F}\right)\)\\\\*
%
Review what these different distributions look like and their relationships between each other.

\subsection{Statistical Models of Solids}

\subsubsection{Basics}
Energy levels in a solid form a band structure.\\*
Review energy level diagrams for metals, insulators, and semiconductors\\\\*
\(n\)-type semiconductors: impurity atoms are \emph{donors} of electrons\\*
\(p\)-type semiconductors: impurity atoms are \emph{acceptors} of electrons\\*
Review energy level diagrams for \(n\) and \(p\)-type semiconductors\\\\*
Effective electron mass: \(\displaystyle m^*=\frac{\hbar^2}{\left(\frac{\mathrm{d}^2E}{\mathrm{d}k^2}\right)}\)

\subsubsection{Einstein's Model of Vibrations in a Solid}
Atoms are treated as SHO's.\\*
Every atom oscillates at the same frequency (Einstein frequency \(\omega_E\)).\\\\*
\(\displaystyle C_V=3Nk_B\left(\frac{\hbar\omega_E}{k_BT}\right)^2\frac{e^{\hbar\omega_E/k_BT}}{\left(e^{\hbar\omega_E/k_BT}-1\right)^2}\)\\\\*
when \(k_BT\gg\hbar\omega_E\rightarrow C_V=3Nk_B\)

\subsubsection{Debye's Model}
Atoms are treated as SHO's.\\*
Atoms oscillate within a range frequencies.\\\\*
Developed by considering the speed of sound in a material: \(\displaystyle \frac{3}{\bar{s}^3}=\frac{1}{\bar{s}_L^3}+\frac{2}{\bar{s}_T^3}\)\\*
\(\bar{s}\) is the average speed of sound, \(L\) means longitudinal, and \(T\) means traverse\\*
Debye frequency: \(\displaystyle\omega_D=\bar{s}\left(\frac{6\pi^2N}{V}\right)^{1/3}\)\\*
Debye energy: \(E_D=\hbar\omega_D\)\\\\*
\(\displaystyle C_V=\frac{2\pi^2k_B^4T^3V}{5\hbar^3\bar{s}^3}\)
