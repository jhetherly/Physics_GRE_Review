\section{Quantum Mechanics}

This section goes well beyond what is necessary for the PGRE.
Focus mainly on energy levels and probability as these are common themes on all the practice tests.
Perturbation theory is usually just one question but is typically straightforward.
Singlet and triplet spin states come up from time to time as well.
Again, any of this (and more) could be on the exam, but it would be unusual for some of the more advanced material to be there.\\*

\subsection{The Schr\"odinger Equation}

\subsubsection{Time-Dependent}
\(i\hbar\dot{\Psi}=H\Psi\)\\*
Solution: \(\Psi(\mathbf{r},t)=\Psi(\mathbf{r},0)e^{-iHt/\hbar}\)

\subsubsection{Time-Independent}
\(H\psi_n=E_n\psi_n\)\\*
Solution: \(\displaystyle\Psi(\mathbf{r},t)=\sum_{n=1}^{m}c_n\psi_n(\mathbf{r},0)e^{-iE_nt/\hbar}\) (\(m\) can go to \(\infty\))\\\\*
These eigenfunctions (\(\psi_n\)) are called stationary states.
Every expectation value is constant in time. (i.e. \(\langle\hat{p}\rangle=0\) because \(\langle \hat{x}\rangle=\mathrm{const.}\))

\subsubsection{Boundary Conditions}
\(\Psi\) and \(\nabla\Psi\) are both continuous.\\*
If \(V(\mathbf{r}_0)\to\pm\infty\) then only \(\Psi\) is continuous at \(\mathbf{r}_0\).

\subsubsection{Normalization}
\(\int_{-\infty}^{\infty}|\Psi(\mathbf{r},t)|^2\,\mathrm{d}\mathbf{r}=1\)

\subsection{General Information}

\subsubsection{de Broglie Wavelength}
For any particle: \(\displaystyle\lambda=\frac{h}{p}\to p=\hbar k\)

\subsubsection{Energy of a Photon}
\(E=h\nu=\frac{hc}{\lambda}\)

\subsubsection{Operators}
Any operator can be decomposed into Hermitian and anti-Hermitian parts:\\*
\(\displaystyle \Omega=\frac{\Omega+\Omega^{\dag}}{2}+\frac{\Omega-\Omega^{\dag}}{2}\)\\\\*
In the \(x\) basis:
\begin{itemize}
\item \(\hat{x}=x\)
\item \(\hat{p}=-i\hbar\frac{\partial}{\partial x}\)
\end{itemize}
\(\hat{L}_z=-i\hbar\frac{\partial}{\partial\phi}\)\\\\*
In the \(p\) basis:
\begin{itemize}
\item \(\hat{x}=i\hbar\frac{\partial}{\partial p}\)
\item \(\hat{p}=p\)
\end{itemize}

\subsubsection{Change of Basis}
\(x\) basis: \(\displaystyle\Psi(x,t)=\frac{1}{\sqrt{2\pi\hbar}}\int_{-\infty}^{\infty}e^{ipx/\hbar}\Phi(p,t)\,\mathrm{d}p\)\\\\*
\(p\) basis: \(\displaystyle\Phi(p,t)=\frac{1}{\sqrt{2\pi\hbar}}\int_{-\infty}^{\infty}e^{-ipx/\hbar}\Psi(x,t)\,\mathrm{d}x\)

\subsubsection{Commutation Relations}
\([A,B]=AB-BA\)\\*
\([A,B]_+=AB+BA\)\\*
\([AB,C]=A[B,C]+[A,C]B\)\\\\*
\([\hat{x},\hat{p}]=i\hbar\)\\*
\(\displaystyle [f(\hat{x}),\hat{p}]=i\hbar\frac{\mathrm{d}f}{\mathrm{d}x}\)\\*
\([\hat{L}_i,\hat{L}_j]=\epsilon_{ijk}i\hbar\hat{L}_k\)\\*
\([\hat{L}^2,\hat{L}_i]=0\)\\*
\([H,\hat{L}_i]=[H,\hat{L}^2]=0\)

\subsubsection{Uncertainty Principle}
Standard Deviation: \(\sigma_A=\sqrt{\langle A^2\rangle-\langle A\rangle^2}\)\\\\*
\(\sigma_A\sigma_B \geq \frac{1}{2}|\langle[A,B]\rangle|\)\\\\*
Common uncertainties:\\*
\(\sigma_x\sigma_p\geq\frac{\hbar}{2}\)\\*
\(\sigma_E\sigma_t\geq\frac{\hbar}{2}\)\\*
\(\sigma_{L_x}\sigma_{L_y}\geq\frac{\hbar}{2}|\langle L_z\rangle|\)

\subsubsection{Ehrenfest's Theorem}
\(\displaystyle \frac{\mathrm{d}}{\mathrm{d}t}\langle\hat{Q}\rangle=\frac{i}{\hbar}\langle[H,\hat{Q}]\rangle+\left<\frac{\partial \hat{Q}}{\partial t}\right>\)\\\\*
Uses:\\*
\(\displaystyle \frac{\mathrm{d}}{\mathrm{d}t}\langle\mathbf{p}\rangle=\langle-\nabla V\rangle\)\\*
\(\displaystyle \frac{\mathrm{d}}{\mathrm{d}t}\langle\mathbf{L}\rangle=\langle\mathbf{r}\times(-\nabla V)\rangle\)
\newpage
\subsubsection{Probability}
Probability Density: \(\int_{-\infty}^{\infty} P(\mathbf{r})\,\mathrm{d}\mathbf{r}=\int_{-\infty}^{\infty} |\Psi(\mathbf{r})|^2\,\mathrm{d}\mathbf{r}\)\\*
Most Probable Value of \(r\): set \(\frac{\mathrm{d}}{\mathrm{d}r}|\psi(r)|^2r^2=0\), then solve for \(r\) (the \(r^2\) comes from \(\mathrm{d}\mathbf{r} = r^2\sin(\theta)\mathrm{d}r\mathrm{d}\theta\mathrm{d}\phi\))\\\\*
Probability Current: \(\displaystyle J(x,t)=\frac{i\hbar}{2m}\left(\psi\frac{\partial\psi^*}{\partial x}-\psi^*\frac{\partial\psi}{\partial x}\right)\)\\*
Probability of finding a particle in the range \(a<x<b\) at time \(t\): \(\displaystyle\frac{\mathrm{d}P_{ab}}{\mathrm{d}t}=J(a,t)-J(b,t)\)\\\\*
\(\displaystyle\langle H\rangle=\sum_{n=1}^{m}|c_n|^2E_n\)\\*
(same \(c_n\) as in \(\displaystyle\sum_{n=1}^{m}c_n\psi_n(\mathbf{r},0)e^{-iE_nt/\hbar}\))\\*
\(|c_n|^2\) tells you the probability that a measurement of the energy would yield the value \(E_n\).\\*
\(\displaystyle\sum_{n=1}^{m}|c_n|^2=1\)

\subsection{Common Solved Problems}
Be sure to study how each of these solutions look like when they are plotted (especially the first two).
Specifically, focus on how many ``nodes'' each eigenfunction has and where they are located.
When an infinite barrier is introduced to a potential only eigenfunctions with a ``node'' at that barrier survive (think ``wave on a string'').

\subsubsection{Infinite Square Well}
Potential: \[V(x) = \left\{
\begin{array}{l l}
  0 & \quad \mbox{\(0<x<a\)}\\
  \infty & \quad \mbox{otherwise}\\ \end{array} \right. \]
Eigenfunctions: \(\psi_n(x)=\sqrt{\frac{2}{a}}\sin(k_nx)\) where \(k_n=\frac{n\pi}{a}\), \(n=1,2,3,\ldots\)\\*
Energy Levels: \(\displaystyle E_n=\frac{\hbar^2k_n^2}{2m}=\frac{\hbar^2\pi^2}{2ma^2}n^2\)

\subsubsection{Harmonic Oscillator}
Potential: \(V(x)=\frac{1}{2}m\omega^2x^2\)\\*
Eigenfunctions: \(\psi_n(x)=\frac{1}{\sqrt{n!}}(a_+)^n\psi_0\) where \(a_+\) is the raising operator and\\*
\(\displaystyle\psi_0(x)=\left(\frac{m\omega}{\pi\hbar}\right)^{1/4}e^{-\frac{m\omega}{2\hbar}x^2}\)\\*
Energy Levels: \(\displaystyle\hbar\omega\left(n+\frac{1}{2}\right)\),  \(n=0,1,2,\ldots\)\\\\*
Raising and Lowering Operators: \(a_{\pm}=\frac{1}{\sqrt{2\hbar m\omega}}(\pm ip+m\omega x)\)\\*
\([a_-,a_+]=1\)\\*
\(H=\hbar\omega\left(a_-a_+-\frac{1}{2}\right)=\hbar\omega\left(a_+a_-+\frac{1}{2}\right)\)\\*
\(a_+\psi_n=\sqrt{n+1}\psi_{n+1}\)\\*
\(a_-\psi_n=\sqrt{n}\psi_{n-1}\)\\*
\(a_-a_+\psi_n=(n+1)\psi_{n}\)\\*
\(a_+a_-\psi_n=n\psi_{n}\)\\*
of course, \(a_-\psi_0=0\) and \(a_+\psi_{n_{highest}}=0\)

\subsubsection{Free Particle}
Potential: \(V(x)=0\)\\*
\(v_{classical}=v_{group}=2v_{phase}\)\\\\*
\(\displaystyle\Psi(x,t)=\frac{1}{\sqrt{2\pi}}\int_{-\infty}^{\infty}\phi(k)e^{i(kx-\frac{\hbar k^2}{2m}t)}\,\mathrm{d}k\)\\*
where \(\displaystyle\phi(k)=\frac{1}{\sqrt{2\pi}}\int_{-\infty}^{\infty}\Psi(x,0)e^{-ikx}\,\mathrm{d}x\)

\subsubsection{Delta-Function Potential}
Potential: \(V(x)=-\alpha\delta(x)\)\\*
Eigenfunction: \(\displaystyle\psi(x)=\frac{\sqrt{m\alpha}}{\hbar}e^{-m\alpha|x|/\hbar^2}\)\\*
Only One Bound State Energy: \(\displaystyle E_0=-\frac{m\alpha^2}{2\hbar^2}\)\\\\*
Reflection \& Transmission Coefficients:\\\\*
\(R+T=1\)\\\\*
\(\displaystyle R=\frac{1}{1+(E/|E_0|)}\)\\*
\(\displaystyle T=\frac{1}{1+(|E_0|/E)}\)

\subsubsection{Finite Square Well}
Potential: \[V(x) = \left\{
\begin{array}{l l}
  -V_0 & \quad \mbox{\(-a<x<a\)}\\
  0 & \quad \mbox{otherwise}\\ \end{array} \right. \]
With a wide, deep well the energies approach those of an infinite square well.\\*
\(\displaystyle E_n+V_0=\frac{\hbar^2k_n^2}{2m}\)\\\\*
With a shallow, narrow well there will always be at least one bound state no matter how weak the well is.
\newpage
\subsubsection{Hydrogen Atom}
Potential: \(\displaystyle V(r)=-\frac{e^2}{4\pi\epsilon_0}\frac{1}{r}\)\\*
The eigenfunctions (\(\psi_{nlm_l}(r,\theta,\phi)\)) are complicated and involve Laguerre polynomials and the spherical harmonics.
However, the ground state of the hydrogen atom is easy to remember.\\*
\(\displaystyle\psi_{100}(r)=\frac{1}{\sqrt{\pi a^3}}e^{-r/a}\) where \(a\) is the Bohr radius (\(a\approx .53\mathrm{\text{\AA}}\))\\*
Energy Levels: \(\displaystyle E_n=-\frac{E_1}{n^2}\) where \(E_1\approx 13.6\mathrm{eV}\)\\*
It is important to know that \(E_1\propto m_eZ_1^2Z_2^2\) where \(m_e\) is the mass of the orbiting body (electron), \(Z_1\) is the charge of the orbiting body (in units of electron charge), and \(Z_2\) is the charge of the central body (nucleus).\\\\*
ETS frequently makes you alter the energy level formula for positronium and helium.
Just replace \(m_e\) in \(E_1\) with the reduced mass \(\mu=\frac{m_e}{2}\) for positronium.
For helium, just remember \(Z_2\to 2\).

\subsection{Angular Momentum}
Orbital: \(\mathbf{L}\times\mathbf{L}=i\hbar\mathbf{L}\) (or \([\hat{L}_i,\hat{L}_j]=\epsilon_{ijk}i\hbar\hat{L}_k\))\\*
This means that one cannot have a completely determined angular momentum \emph{vector} just as one cannot completely determine both position and momentum.\\\\*
Spin: \(\mathbf{S}=\frac{\hbar}{2}\vec{\sigma}\)\\\\*
Pauli matrices: \(\sigma_x= \left[\!
  \begin{array}{ c c }
     0 & 1 \\
     1 & 0
  \end{array} \!\right]
\),
\(\sigma_y= \left[\!
  \begin{array}{ c c }
     0 & -i \\
     i & 0
  \end{array} \!\right]
\),
\(\sigma_z= \left[\!
  \begin{array}{ c c }
     1 & 0 \\
     0 & -1
  \end{array} \!\right]
\)\\\\*
It is convenient to express spin in terms of up/down vectors:\\*
Up: \(|\!\uparrow\rangle=\left[\!\begin{array}{c}1 \\ 0 \end{array}\!\right]\)\\*
Down: \(|\!\downarrow\rangle=\left[\!\begin{array}{c}0 \\ 1 \end{array}\!\right]\)\\*
\(\mathbf{S}\times\mathbf{S}=i\hbar\mathbf{S}\)\\\\*
Total: \(\mathbf{J}=\mathbf{L}+\mathbf{S}\)\\*
\(\mathbf{J}\times\mathbf{J}=i\hbar\mathbf{J}\)

\subsubsection{Raising and Lowering Operators}
\(\hat{L}_\pm=\hat{L}_x\pm i\hat{L}_y\)\\*
\([\hat{L}_z,\hat{L}_\pm]=\pm\hbar\hat{L}_\pm\)

\subsubsection{Eigenvalues}
\(\hat{L}^2|lm_l\rangle=l(l+1)\hbar^2|lm_l\rangle\), \(l=0,1,2,\ldots,n\)\\*
\(\hat{L}_z|lm_l\rangle=m_l\hbar|lm_l\rangle\), \(m_l=-l,-l+1,\ldots,0,\ldots,l-1,l\)\\*
\(\hat{L}_\pm|lm_l\rangle=A_l^{m_l}\hbar^2|l(m_l\pm 1)\rangle\)\\\\*
\(\hat{S}^2|sm_s\rangle=s(s+1)\hbar^2|sm_s\rangle\), \(s=0,1,2,\ldots\)\\*
\(\hat{S}_z|sm_s\rangle=m_s\hbar|sm_s\rangle\), \(m_s=-s,-s+1,\ldots,0,\ldots,s-1,s\)

\subsubsection{Addition of Angular Momentum}
\(s=1\) (triplet states):\\*
\(|11\rangle =\) \( \uparrow\uparrow\)\\*
\(|10\rangle =\) \(\frac{1}{\sqrt{2}}(\uparrow\downarrow+\downarrow\uparrow)\)\\*
\(|1(-1)\rangle =\) \(\downarrow\downarrow\)\\\\*
\(s=0\), \(m_s=0\) (singlet state):\\*
\(|00\rangle =\) \(\frac{1}{\sqrt{2}}(\uparrow\downarrow-\downarrow\uparrow)\)

\subsection{Time-Independent Perturbation Theory}
\(H=H_0+\lambda\Delta H\) where \(H_0\) is a solvable Hamiltonian with basis functions \(|n^{(0)}\rangle\)\\*
\(E_n=E_n^{(0)}+\lambda E_n^{(1)}+\ldots\)\\*
\(|n\rangle=|n^{(0)}\rangle+\lambda |n^{(1)}\rangle+\ldots\)

\subsubsection{First-Order Energy Correction}
\(E_n^{(1)}=\langle n^{(0)}|\Delta H|n^{(0)}\rangle\)

\subsubsection{First-Order Eigenfunction Correction}
\(\displaystyle |n^{(1)}\rangle=\sum_{k\neq n}\frac{\langle k^{(0)}|\Delta H|n^{(0)}\rangle}{E_n^{(0)}-E_k^{(0)}}|k^{(0)}\rangle\)\\*
The key point of this equation is \(\langle k^{(0)}|\Delta H|n^{(0)}\rangle\), which determines what new eigenfunctions will be zero (typically using even/odd symmetry arguments).
