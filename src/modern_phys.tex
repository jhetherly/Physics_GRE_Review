\section{Modern Physics}
This is a large section and covers a variety of subjects that the PGRE splits up into multiple topics.
These sections try to stick to only what is covered on the practice tests because they are very deep topics.
In total, this section is worth about 25\% of the problems on the test.

\subsection{Special Relativity}
Be sure to understand proper length and time.
Visualizing the different reference frames is also extremely helpful.

\subsubsection{Postulates}
\begin{enumerate}
\item The laws of physics must be the same in all inertial reference frames.
\item The speed of light in vacuum has the same value in all inertial reference frames.
\end{enumerate}

\subsubsection{Basics}
\(\displaystyle\beta=\frac{v}{c}\le1\)\\*
Lorentz factor: \(\displaystyle\gamma=\frac{1}{\sqrt{1-\beta^2}}\ge1\)\\*
Invariance of the space-time interval: \(c\mathrm{d}t'-\mathrm{d}x'-\mathrm{d}y'-\mathrm{d}z'=c\mathrm{d}t-\mathrm{d}x-\mathrm{d}y-\mathrm{d}z\)

\subsubsection{Length Contraction}
\(\displaystyle L'=\frac{L_{proper}}{\gamma}\)

\subsubsection{Time Dilation}
\(t'=\gamma t_{proper}\)

\subsubsection{Relativistic Doppler Effect \& Redshift}
Doppler factor: \(\displaystyle \frac{\lambda_{observer}}{\lambda_{source}}=\sqrt{\frac{1\pm\beta}{1\mp\beta}}\)\\*
Redshift: \(\displaystyle z=\frac{\lambda_{observer}-\lambda_{source}}{\lambda_{source}}=\sqrt{\frac{1\pm\beta}{1\mp\beta}}-1\)\\\\*
choose the top set of signs when the observer and source are moving \emph{away} from each other, choose the bottom set of signs when they are moving \emph{toward} each other

\subsubsection{Momentum and Energy}
\(p=\frac{E}{c}\) for massless particles\\*
\(\mathbf{p}=\gamma m\mathbf{v}\) for massive particles\\*
In general: \(p=\frac{1}{c}\sqrt{E^2-\left(mc^2\right)^2}\)\\*
Newton's second law is still valid in the form: \(\displaystyle \mathbf{F}=\frac{\mathrm{d}\mathbf{p}}{\mathrm{d}t}\)\\\\*
%
\(E=\sqrt{\left(mc^2\right)^2+\left(pc\right)^2}=\gamma mc^2=K+mc^2\)\\*
\(K=E-mc^2=mc^2\left(\gamma-1\right)\)\\*
\(p=\sqrt{\left(\frac{K}{c}\right)^2+2mK}\)\\*
Mass-energy relationship: \(\displaystyle E_i=\frac{m_ic^2}{\sqrt{1-\left(\frac{u_i}{c}\right)^2}}=\gamma_im_ic^2\)

\subsubsection{Lorentz Transformation}
For motion along the \(x\)-axis:\\*
\(\lambda=\left[\!
  \begin{array}{ c c c c }
     \gamma & -\beta\gamma & 0 & 0 \\
     -\beta\gamma & \gamma & 0 & 0 \\
     0 & 0 & 1 & 0 \\
     0 & 0 & 0 & 1
  \end{array} \!\right]\)\\*
 \(\lambda^{-1}=\left[\!
  \begin{array}{ c c c c }
     \gamma & \beta\gamma & 0 & 0 \\
     \beta\gamma & \gamma & 0 & 0 \\
     0 & 0 & 1 & 0 \\
     0 & 0 & 0 & 1
  \end{array} \!\right]\)\\*
\begin{eqnarray}
\mathbf{x}'&=&\lambda\mathbf{x} \nonumber\\
ct'&=&\gamma\left(ct-\beta x\right) \nonumber\\
x'&=&\gamma\left(-\beta ct+x\right) \nonumber\\
y'&=&y \nonumber\\
z'&=&z \nonumber
\end{eqnarray}

\subsubsection{Velocity Addition}
If \(v\) is along \(x\)-axis:\\*
\begin{eqnarray}
\displaystyle u_x'&=&\frac{u_x-v}{1-\frac{u_xv}{c^2}}\nonumber\\
\displaystyle u_y'&=&\frac{u_y}{\gamma\left(1-\frac{u_xv}{c^2}\right)}\nonumber\\
\displaystyle u_z'&=&\frac{u_z}{\gamma\left(1-\frac{u_xv}{c^2}\right)}\nonumber
\end{eqnarray}
To go from the primed coordinates to unprimed:\\*
\(\displaystyle u_x=\frac{u_x'+v}{1+\frac{u_x'v}{c^2}}\)\\\\*
Although you should memorize these, deriving them from the Lorentz transform isn't difficult (but maybe too time consuming for the PGRE):
\begin{eqnarray}
\displaystyle u_x'&=&\frac{\mathrm{d}x'}{\mathrm{d}t'} \nonumber\\
\displaystyle \mathrm{d}x'&=&\gamma\left(\mathrm{d}x-v\mathrm{d}t\right) \nonumber\\
\displaystyle \mathrm{d}t'&=&\gamma\left(\mathrm{d}t-\frac{v}{c^2}\mathrm{d}x\right) \nonumber\\
\displaystyle u_x'&=&\frac{\mathrm{d}x-v\mathrm{d}t}{\mathrm{d}t-\frac{v}{c^2}\mathrm{d}x} \nonumber\\
\displaystyle &=& \frac{(\mathrm{d}x/\mathrm{d}t)-v}{1-\frac{v}{c^2}(\mathrm{d}x/\mathrm{d}t)} \nonumber\\
\displaystyle &=& \frac{u_x-v}{1-\frac{u_xv}{c^2}}\nonumber
\end{eqnarray}
Derivations for \(u_y'\), \(u_z'\), and going from the primed to unprimed coordinates follow similar logic.

\subsubsection{Completely Inelastic Collisions}
Suppose \(m_1\) and \(m_2\) collide inelastically to form \(m_3\):\\\\*
\(\displaystyle m_3=\frac{\gamma_1 m_1+\gamma_2 m_2}{\gamma_3}\) \\*
\(\displaystyle \Delta m=m_3-(m_1+m_2)=\frac{K_1+K_2-K_3}{c^2}\)\\*
These are easily derived from conservation of energy.

\subsection{Atomic Physics}

\subsubsection{Notation}
Know how to write out the electron orbitals in \((nl)^N\) notation.\\*
E.g. \(Z=11\) for sodium. The orbitals for the ground state are \((1s)^2(2s)^2(2p)^6(3s)^1\)\\*
\(s\)=sharp: \(l=0\) (can hold 2 electrons)\\*
\(p\)=principle: \(l=1\) (can hold 6 electrons)\\*
\(d\)=diffuse: \(l=2\) (can hold 10 electrons)\\*
\(f\)=fundamental: \(l=3\) (can hold 14 electrons)\\*
(remember \((4s)\) comes before \((3d)\))\\\\*
ETS also likes the \(^{2S+1}L_J\) notation for atoms. Hund's rules are employed in filling this out correctly.
Taking time to learn and \emph{practice} this is worthwhile because these questions are typically easy enough to do in thirty seconds or less.\\\\*
\(L\)= total orbital angular momentum\\*
\(S\)= total spin\\*
\(J\)= grand total angular momentum\\*
Hund's Rules:
\begin{enumerate}
\item Considering the Pauli principle, the state with the highest spin (\(+\frac{1}{2}\)) has the lowest energy.
\item Considering the Pauli principle, the state with the highest \(L\) has the lowest energy.
\item If a shell is more than half filled, use \(J=S+L\), otherwise use \(J=|S-L|\).
\end{enumerate}
Also, filled shells don't count when constructing this notation.

\subsubsection{Energy and Wavelength of Emitted Photons}
For hydrogen-like atoms:\\*
Energy: \(\displaystyle E_{\gamma}=E_1\left(\frac{1}{n_f^2}-\frac{1}{n_i^2}\right)\) where \(E_1\approx 13.6\mathrm{eV}\)\\\\*
Wavelength: \(\displaystyle \frac{1}{\lambda}=R\left(\frac{1}{n_f^2}-\frac{1}{n_i^2}\right)\) where \(R\approx 10^7 m^{-1}\)\\\\*
Lyman series: \(n_f=1\) (ultraviolet)\\*
Balmer series: \(n_f=2\) (visible)\\*
Pashen series: \(n_f=3\) (infrared)\\*
Look at the hydrogen atom subsection in the Quantum Mechanics section to see how to alter \(E_1\) and \(R\) for positronium and helium (or other elements with higher \(Z_2\)).\\\\*
For heavy elements: \(\displaystyle E_n=\frac{-13.6(eV)Z_{eff}^2}{n^2}\), where \(Z_{eff}\) is the \emph{effective} charge the electron sees.\\\\*
%
Emitted X-Rays:\\*
K shell (\(n=1\)): \(\displaystyle E_K\approx-13.6\left(Z-1\right)^2\)\\*
L shell (\(n=2\)): \(\displaystyle E_L\approx-13.6\frac{\left(Z-1\right)^2}{4}\)\\*
M shell (\(n=3\)): \(\displaystyle E_M\approx-13.6\frac{\left(Z-9\right)^2}{9}\)\\*
\(K_\alpha\) line is from \(L\rightarrow K\): \(E_K-E_L\)\\*
\(K_\beta\) line is from \(M\rightarrow K\): \(E_K-E_M\)\\*
\(L_\alpha\) line is from \(M\rightarrow L\): \(E_L-E_M\)

\subsubsection{Bohr Model}
Radius: \(a_n\approx.53\mathrm{\text{\AA}} \left(\frac{n^2}{m_eZ}\right)\)\\*
Energy: \(E_n\approx13.6eV\left(\frac{m_eZ^2}{n^2}\right)\)

\subsubsection{Ionization}
An ion is an atom with one or more extra/missing electrons.
Knowing this, it is easy to construct a general formula for the ionization energy of an atom with atomic number \(Z_a\):\\*
\(E_{total}=E_{1^{st}}+E_{2^{nd}}+\ldots+E_{n^{th}}\), where \(E_{n^{th}}\) is the \(n^{th}\) ionization energy\\*
It is important to note that if one removes \(Z_a-1\) electrons from an atom, the formula for the energy of the electron is the same as hydrogen except that \(Z_2=Z_a\) (\(Z_2\) is defined in the quantum mechanics section).

\subsubsection{Selection Rules}
Electric dipole transitions:\\*
\(\Delta l=\pm1\)\\*
\(\Delta m_l=0,\pm1\)\\*
\(\Delta j=0,\pm1\)\\*
\(\Delta m_s=0\)

\subsubsection{Gyromagnetic Ratio}
Magnetic Moment: \(\vec{\mu}=\gamma\mathbf{S}\)\\*
\(\displaystyle\gamma = \frac{qg}{2m}\) where \(q\) is the charge and \(g\) is the Lande \(g\)-factor

\subsection{Energy States \& Spectra of Molecules}
\(E=E_{el}+E_{trans}+E_{rot}+E_{vib}\)\\\\*
For diatomic molecules:\\*
\(E_{rot}=\frac{\hbar^2}{2I}J(J+1)\)\\*
\(\Delta E_{rot}=E_J-E_{J-1}=\frac{\hbar^2}{I}J\)\\*
\(E_{vib}=(n+\frac{1}{2})\hbar\omega\), where \(n=0,1,2, \ldots\), \(\omega = \sqrt{\frac{k}{\mu}}\), and \(\mu\) is the reduced mass of the molecule\\*
\(\Delta E_{vib}=\hbar\omega\)

\subsection{Radioactivity}
\(\displaystyle \frac{\mathrm{d}N}{\mathrm{d}t}=-\lambda N\rightarrow N=N_0 e^{-\lambda t}\), where \(\lambda\) is the decay constant and \(N\) is the number of particle left\\*
Decay rate (activity): \(\displaystyle \left|\frac{\mathrm{d}N}{\mathrm{d}t}\right|=R=\lambda N\rightarrow R=R_0 e^{-\lambda t}\)\\*
Half-life: \(\displaystyle t_{1/2}=\frac{\ln{2}}{\lambda}\rightarrow N=\frac{N_0}{2}\)

\subsection{Nuclear Physics}

\subsubsection{Radius of Nucleus}
\(r=r_0A^{1/3}\), where \(A\) is the number of nucleons and \(r_0=1.2(fm)\)

\subsubsection{Strong Force}
This is the strongest of the four fundamental forces.
It is independent of charge, very short range, and its magnitude depends on the relative spin orientations.

\subsubsection{Nuclear Magneton}
\(\mu_m\equiv\frac{e\hbar}{2m_p}\approx5\times10^{-27}(J/T)\)

\subsubsection{Fission}
Fission is the process whereby a large nucleus is split into smaller pieces (other nuclei and subatomic particles).
This process releases a large amount of energy (disintegration energy).\\*
Disintegration energy: \(\displaystyle Q=\left(M_n-\left(\sum_iM_i\right)\right)c^2=\Delta mc^2\), where \(M_n\) is the mass of the nucleus before the split and \(M_i\) is the mass of product \(i\)

\subsubsection{Fusion}
Fusion is the process of smashing atoms and/or particles together to create heavier nuclei.
This releases even more energy (per product) than fission.
The energy released is the binding energy of the resultant nucleus.\\\\*
Binding energy: \(\displaystyle E_B=\sum_i{m_ic^2}-Mc^2\), where \(m_i\) is the mass of the free component atom/particle \(i\) and \(M\) is the mass of the bound system

\subsection{Particle Physics}

\subsubsection{Types of Particles}
Hadrons: particles that interact through the strong force\\*
Examples:\\*
Mesons: zero or integer spin (pions (\(\pi^+\), \(\pi^-\), and \(\pi^0\)) all have zero spin)\\*
Baryons: half-integer spin (protons and neutrons have half-integer spin)\\\\*
Leptons: particles that do not interact by means of the strong force\\*
Only twelve exist: \(e\), \(\mu\), \(\tau\), \(\nu_e\), \(\nu_{\mu}\), \(\nu_{\tau}\) (and their anti-particle counterparts)

\subsubsection{Alpha Decay}
\(\,^A_ZX\rightarrow \,^{A-4}_{Z-2}Y+\,^4_2He\)

\subsubsection{Beta Decay}
\(\,^A_ZX\rightarrow \,^{\hspace{.35cm}A}_{Z+1}Y+e^-+\bar{\nu}_e\) (electron and anti-electron-neutrino)\\*
\(\,^A_ZX\rightarrow \,^{\hspace{.35cm}A}_{Z-1}Y+e^++\nu_e\) (positron and electron-neutrino)\\\\*
%
Examples:\\*
\(n\rightarrow p+e^-+\bar{\nu}_e\)\\*
\(p\rightarrow n+e^++\nu_e\)

\subsubsection{Gamma Decay}
\(\,^A_ZX^*\rightarrow \,^A_ZX+\gamma\)\\\\*
%
Example (two-part decay):\\*
\(\,^{12}_{\hspace{.15cm}5}B\rightarrow \,^{12}_{\hspace{.15cm}6}C^*+e^-+\bar{\nu}_e\)\\*
\(\,^{12}_{\hspace{.15cm}6}C^*\rightarrow \,^{12}_{\hspace{.15cm}6}C+\gamma\)

\subsubsection{Particle Decay}
\(\pi^-\rightarrow\mu+\bar{\nu}_{\mu}\)\\*
\(\mu\rightarrow e+\bar{\nu}_e+\nu_e\)

\subsubsection{Neutron Capture}
\(\,^1_0n+\,^A_ZX\rightarrow\,^{A+1}_{\hspace{.35cm}Z}X^*\rightarrow\,^{A+1}_{\hspace{.35cm}Z}X+\gamma\)

\subsubsection{Pair-Production}
A \(\gamma\)-ray photon with sufficiently high energy interacts with a nucleus, and an electron-positron pair is created.\\*
\(E_{\gamma}\ge1(MeV)\)\\*
Due to conservation of momentum, two \(\gamma\)-rays are created at annihilation:\\*
\(e^++e^-\rightarrow2\gamma\)

\subsubsection{Conservation Laws}
Baryon number:\\*
\(+1\) for baryons\\*
\(-1\) for anti-baryons\\*
\(0\) for all others\\\\*
Lepton number:\\*
\(+1\) for  \(e\), \(\mu\), \(\tau\), \(\nu_e\), \(\nu_{\mu}\), \(\nu_{\tau}\)\\*
\(-1\) for  \(\bar{e}\), \(\bar{\mu}\), \(\bar{\tau}\), \(\bar{\nu}_e\), \(\bar{\nu}_{\mu}\), \(\bar{\nu}_{\tau}\)\\*
\(0\) for all others\\\\*
Strangeness:\\*
In a nuclear reaction or decay that occurs via the strong force, strangeness is conserved.\\*
In processes that occur via the weak interaction, strangeness may not be conserved.

\subsection{Devices}

\subsubsection{The Laser}
LASER stands for: Light Amplification by Stimulated Emission of Radiation
\begin{itemize}
\item The emitted light is coherent (same phase)
\item The emitted light is nearly monochromatic (one wavelength)
\item Minimal divergence
\item Highest intensity of any light source
\end{itemize}
The majority of an assembly of atoms is brought to an excited state through ``population inversion.''\\*
``Population inversion'' can be achieved through ``optical pumping'' where atoms are exposed to a given wavelength of light.
This wavelength is enough to excite the atoms just above the metastable level.
The atoms rapidly lose energy and fall to the metastable level.
``Induced emission'' occurs when the atom goes from the metastable state to the ground state (this is what produces the light).

\subsubsection{Michelson Interferometer}
This devise takes advantage of the difference in path length by two different beams of light.\\*
For a basic interferometer, the equations for constructive and destructive interference are:
\begin{itemize}
\item Constructive: \(2\Delta d=m\lambda\) where \(m=0,\pm1,\pm2,\ldots\)
\item Destructive: \(2\Delta d=(m+\frac{1}{2})\lambda\) where \(m=0,\pm1,\pm2,\ldots\)
\end{itemize}
\(\Delta d\) is the distance the moveable arms travels and \(m\) is the number of fringes produced.
These equations are easily derived by looking at the path difference for the two beams of light and noting the condition for constructive and destructive interference.
ETS likes to modify the apparatus slightly, but the general concept remains the same.
Typically, when the number of fringes is specified one typically needs the constructive interference equation.

\subsection{Important Effects}

\subsubsection{Photoelectric Effect}
Important result: Light incident on a metallic surface causes electrons to be emitted from the surface with a kinetic energy of \(K_{max}=e\Delta V_s=h\nu-\phi\), where \(\phi\) is the work function of the metal (represents the minimum energy with which an electron is bound to the metal).\\*
When \(\Delta V<0\) there is a ``stopping voltage (\(\Delta V_s\))'' where the electrons haven't enough energy to overcome the potential, hence no current is established.
\begin{itemize}
\item Classically: electrons should absorb energy continuously and kinetic energy should rise with light intensity\\*
Observed: \(K_{max}\) depends on \(\Delta V_s\)
\item Classically: at low intensities a buildup time should be observed\\*
Observed: almost instantaneous emission of electron even at very low intensities
\item Classically: electron should be ejected for all frequencies and only depend on intensity\\*
Observed: no electrons are emitted below a certain cutoff frequency (\(f_c=\frac{\phi}{h}\))
\item Classically: kinetic energy should only depend on intensity, not frequency of light\\*
Observed: \(K_{max}\) increases with light frequency
\end{itemize}
Review the schematic for this device.

\subsubsection{Compton Effect}
Describes the shift in wavelength for light scattered from particles.\\*
\(\Delta\lambda=\frac{h}{cm}\left(1-\cos{\theta}\right)=\lambda_C\left(1-\cos{\theta}\right)\), where \(\theta\) is the scattering angle and \(\lambda_C\) is the Compton wavelength

\subsubsection{Spectrum Line-Splitting}

Zeeman effect: When you apply a uniform external magnetic field, each transition energy (\(E_{n_1,l_1\rightarrow n_2,l_2}\)) it splits into three equally-spaced lines, due to whether \(m_l\) increases by one, decreases by one, or stays the same in the transition.\\\\*
Anomalous Zeeman effect: In the Zeeman effect, the contribution of electron spin to the total angular momentum means that there aren't always three lines and they are not always equally spaced.\\\\*
Stark effect: When you apply a uniform electric field, it induces a dipole moment in the atoms and the field in turn interacts with the dipole moment. The effect depends on \(|m_j|\). If \(j\) is an integer, it splits into \(j+1\) levels. If \(j\) is a half-integer, it splits into \(j+\frac{1}{2}\) levels.\\\\*
Stern-Gerlach experiment: Atoms are sent through a nonuniform magnetic field and are split into \(2S+1\) beams, where \(S\) is the spin of the atom.\\*
This experiment verified space quantization exists (spin).

\subsubsection{X-Ray Spectra}
``Auger transition'' (internal conversion): When an incoming particle knocks out an inner-shell electron (and that vacancy gets filled by an outer-shell electron), a spike in the spectrum is created\\*
``Bremsstrahlung'' (braking radiation): This is the continuos spectrum of light released by the deceleration of an electron.\\\\*
Together, these effects create a spectrum that is continuos with a few spikes.

\subsubsection{Light-Matter Interaction Energy Levels}
Low-energy: Photoelectric effect\\*
Mid-energy: Compton effect\\*
High-energy: Pair-production (\(\gamma\rightarrow e^-+e^+\))

\subsubsection{Superconductivity}
A superconductor is conductor with no resistance to the flow of electric current.
It is a perfect diamagnet having a negative magnetic susceptibility.
The magnetic flux in a superconductor cannot change (\(\frac{\partial\Phi_B}{\partial t}=0\)).\\*
Meissner effect: a superconductor repels a permanent magnet.

\subsection{Cosmology}

\subsubsection{Hubble's Law}
\(v=HR\)\\*
Hubble's constant: \(H\approx17\times10^{-3} (m/s\cdot ly)\)\\*
\(v\) is the velocity of the galaxy\\*
\(R\) is the distance from Earth

\subsubsection{Black Holes}
Radius of a black hole: \(\displaystyle R=\frac{2GM}{c^2}\)\\\\*
You may realize that this is identical to the Newtonian escape velocity formula with \(v=c\).
By all accounts, this is dumb luck that it can be related to Newtonian gravity as space-time around a black hole must be handled with general relativity.
However, it is helpful to remember this relation as most (all?) students haven't formally studied general relativity yet.
