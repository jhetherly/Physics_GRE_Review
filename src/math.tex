\section{Useful Mathematical Information}
Here are some helpful mathematical notes.
I also put more mathematical information throughout this document in various section that is not contained here, so be sure to study that information as well.

\subsection{Numerical Data}
\subsubsection{Mathematical Data}
\(\pi\approx3.1\)\\*
\(e\approx2.7\)\\*
\(\mathrm{ln}2\approx.7\)\\*
\(\sqrt{2}\approx1.4\)\\*
\(\sqrt{3}\approx1.7\)\\*
\(\sqrt{10}\approx\pi\)\\\\*
\(\sin(30^{\circ})=\cos(60^{\circ})=\frac{1}{2}=.5\)\\*
\(\sin(45^{\circ})=\cos(45^{\circ})=\frac{1}{\sqrt{2}}\approx.71\)\\*
\(\sin(60^{\circ})=\cos(30^{\circ})=\frac{\sqrt{3}}{2}\approx.87\)
\subsubsection{Physical Data}
Gravitational Constant: \(G\approx6.67\times10^{-11}(Nm^2/kg^2)\)\\*
Proton Mass:  \(m_p\approx1.7\times10^{-27}(kg)\)\\*
Electron Mass: \(m_e\approx9.1\times10^{-31}(kg)\)\\*
Electron Charge: \(e\approx1.6\times10^{-19}(C)\)\\*
Vacuum Permittivity: \(\epsilon_0\approx9\times10^{-12}(C^2/Nm^2)\)\\*
Coulomb's Constant: \(k_e=\frac{1}{4\pi\epsilon_0} \approx 9\times10^8(Nm^2/C^2)\)\\*
Vacuum Permeability: \(\mu_0=4\pi\times10^{-7}(N/A^2)\)\\*
Plank's Constant: \(h\approx6.6\times10^{-34}(Js)\)\\*
Modified Plank's Constant: \(\hbar =\frac{h}{2\pi}\approx10^{-34}(Js)\)\\*
Stefan-Boltzman constant: \(\sigma=5.7\times10^{-8}(W/m^2K^4)\)\\*
Speed of Light: \(c=\frac{1}{\sqrt{\epsilon_0\mu_0}}\approx3\times10^{8}(m/s)\)\\*
\\*
Earth Data:\\*
 Acceleration Due to Gravity: \(g \approx 10(m/s^2)\)\\*
 Year: \(T_{year}\approx\pi\times10^{7}(s)\)\\*
Average Radius: \(R_E\approx6\times10^6 (m)\)\\*
Mass: \(M_E\approx6\times10^{24}(kg)\)\\*
Average distance from the Sun to Earth: \(1(A.U.)\approx1.5\times10^{11}(m)\)\\*
Average distance from the Moon to Earth: \(\sim4\times10^{8}(m)\)\\*
Intensity at Earth's surface: \(\sim1.3\times10^3(W/m^2)\)\\*
Atmospheric Pressure: \(1(atm)\approx10^5(Pa)\)\\*
Mass of Atmosphere: \(\sim5\times10^{18}(kg)\)\\*
Density of Atmosphere at Sea Level: \(\sim1.2(kg/m^3)\)\\*
Number Density of Atmosphere at Sea Level: \(\sim2.5\times10^{25}(molecules/m^3)\)\\*
\(90\%\) of atmosphere is below \(16(km)\)

\subsection{Areas and Volumes}

\subsubsection{Areas}
Circle: \(\pi r^2\)\\*
Triangle: \(\frac{1}{2}bh\)\\*
Function in \(x\)-\(y\) Plane: \(\int_{x_1}^{x_2}f(x)\,\mathrm{d}x\)\\*
Sphere: \(4\pi r^2\)\\*
Cylinder: \(2\pi r^2+2\pi rl\)

\subsubsection{Volumes}
Sphere: \(\frac{4}{3}\pi r^3\)\\*
Cylinder: \(\pi r^2l\)

\subsection{Trigonometric Identities}

\subsubsection{Pythagorean Identity}
\(\sin^2(\theta)+\cos^2(\theta)=1\)

\subsubsection{Double Angle}
\(\sin(2\theta)=2\sin(\theta)\cos(\theta)\)\\*
\(\cos(2\theta)=\cos^2(\theta)-\sin^2(\theta)\)

\subsubsection{Half Angle}
\(\sin^2\left(\frac{\theta}{2}\right)=\frac{1}{2}\left(1-\cos(\theta)\right)\)\\*
\(\cos^2\left(\frac{\theta}{2}\right)=\frac{1}{2}\left(1+\cos(\theta)\right)\)

\subsubsection{Euler's Identity}
\(e^{i\theta}=\cos{\theta}+i\sin{\theta}\)

\subsection{Vector Identities}

\subsubsection{Triple Scalar Product}
\(\mathbf{A}\cdot(\mathbf{B}\times\mathbf{C})=\mathbf{B}\cdot(\mathbf{C}\times\mathbf{A})=\mathbf{C}\cdot(\mathbf{A}\times\mathbf{B})\)

\subsubsection{Triple Vector Product}
\(\mathbf{A}\times(\mathbf{B}\times\mathbf{C})=\mathbf{B}(\mathbf{A}\cdot\mathbf{C})-\mathbf{C}(\mathbf{A}\cdot\mathbf{B})\)

\subsection{Fundamental Theorem of Caculus}
\(\displaystyle \int_{x_1}^{x_2}{\frac{\mathrm{d}f(x)}{\mathrm{d}x}\mathrm{d}x}=f(x_2)-f(x_1)\)\\\\*
\(\displaystyle \int_{\mathbf{a}}^{\mathbf{b}}{(\nabla f)\cdot\mathrm{d}\mathbf{l}}=f(\mathbf{b})-f(\mathbf{a})\)\\\\*
\(\displaystyle \int_{A}{(\nabla\times\mathbf{F})\cdot\mathrm{d}\mathbf{a}}=\oint_{\delta A}{\mathbf{F}\cdot\mathrm{d}\mathbf{l}}\)\\\\*
\(\displaystyle \int_{V}{(\nabla\cdot\mathbf{F})\mathrm{d}V}=\oint_{\delta V}{\mathbf{F}\cdot\mathrm{d}\mathbf{a}}\)\\\\*
In general: \(\displaystyle\int_{\Omega}{\mathrm{d}\omega}=\oint_{\delta\Omega}{\omega}\)\\*
Here \(\omega\) is a (\(n-1\))-form, \(\Omega\) is a manifold of dimension \(n\), and \(\mathrm{d}\) is the exterior derivative

\subsection{Fourier Series}
For function with \(2L\) periodicity:\\*
\(\displaystyle f(x)=\frac{1}{2}a_0+\sum_{n=1}^\infty a_n\cos(nx)+\sum_{n=1}^\infty b_n\sin(nx)\)\\*
\(\displaystyle a_0=\frac{1}{L}\int_{-L}^{L}f(x)\,\mathrm{d}x\)\\*
\(\displaystyle a_n=\frac{1}{L}\int_{-L}^{L}f(x)\cos(nx)\,\mathrm{d}x\)\\*
\(\displaystyle b_n=\frac{1}{L}\int_{-L}^{L}f(x)\sin(nx)\,\mathrm{d}x\)\\\\*
If the function has half- or quarter-wave symmetry, then \(n\) takes on only odd values.

\subsection{Delta Function}
\[\displaystyle\int_{a}^{b}f(x)\delta(x-x_0)\,\mathrm{d}x = \left\{
\begin{array}{l l}
  f(x_0) & \quad \mbox{\(a\leq x_0\leq b\)}\\
  0 & \quad \mbox{otherwise}\\ \end{array} \right. \]
\(\delta(cx)=\frac{1}{|c|}\delta(x)\)

\subsection{Step Function}
\[\theta(x) = \left\{
\begin{array}{l l}
  1 & \quad \mbox{\(x>0\)}\\
  1/2 & \quad \mbox{x=0}\\
  0 & \quad \mbox{\(x<0\)}\\ \end{array} \right. \]
\(\displaystyle\frac{\mathrm{d}\theta}{\mathrm{d}x}=\delta(x)\)

\subsection{Legendre Polynomials}
\(\displaystyle P_l(x)=\frac{1}{2^ll!}\left(\frac{\mathrm{d}}{\mathrm{d}x}\right)^l\left(x^2-1\right)^l\)\\\\*
\(P_l(1)=1\)\\*
\(P_0(x)=1\)\\*
\(P_1(x)=x\)\\*
\(P_2(x)=(3x^2-1)/2\)\\\\*
\(\displaystyle \int_{-1}^{1}P_l(x)P_{l'}(x)\mathrm{d}x=\frac{2}{2l+1}\delta_{ll'}\)

\subsection{Spherical Harmonics}
\(\displaystyle Y^{m}_{l}\left(\theta,\phi\right)=\sqrt{\frac{(2l+1)(l-m)!}{4\pi(l+m)!}}P_{l}^{m}(\cos{\theta})e^{im\phi}\)\\*
where \(P_{l}^{m}\) is the associated Legendre polynomial\\\\*
\(\displaystyle\int_{0}^{\pi}{\int_{0}^{2\pi}{Y_{l}^{m}Y_{l'}^{*m'}\sin{\theta}\mathrm{d}\phi\mathrm{d}\theta}}=\delta_{ll'}\delta_{mm'}\)\\\\*
The concept of an orthonormal set of basis vectors is very important for this test.

\subsection{Common Approximations}
For physical units:\\*
\(1(mi)\approx1.6(km)\)\\*
\(1(rpm)=\frac{\pi}{30}(rad/s)\approx.1(rad/s)\)\\\\*
For small \(\beta\) (special relativity):\\*
\(\gamma\approx1+\frac{1}{2}\beta^2\)\\*
\(\frac{1}{\gamma}\approx1-\frac{1}{2}\beta^2\)\\\\*
For small \(x\):\\*
\(\cos{x}\approx1\)\\*
\(\tan{x}\approx\sin{x}\approx x\)\\*
\(\sqrt{1+x}\approx1+\frac{1}{2}x\)\\*
\(e^x\approx1+x\)\\*
\((1+x)^n\approx1+nx\)
