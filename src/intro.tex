\section{Introduction}

This document is intended for those studying for the GRE subject test in physics.
It should be used alongside various undergraduate texts as a sort of guide and it does not contain any sample problems.
As such, the available practice exams (four as of the writing of this document: GR8677, GR9277, GR9677, and GR0177) and the web site \texttt{http://grephysics.net/} are invaluable resources in preparing for the exam.
Another great resource is the Ohio State SPS website.
They have ``minitests'' that are categorized by subject so you can practice certain subjects individually and links to the available practice exams.
The vast majority of the Physics GRE (or PGRE as it will be referred to from now on) questions are sophomore and junior physics undergraduate level (in other words, one should be able to answer most of the questions on the exam by end of the junior year).
A two month study period should be sufficient for most physics students to make an adequate score.
This gives enough time to review and study the material as well as practice the exams and refine the student's number crunching ability.

The four PGRE tests are vital to understanding what could be asked on future tests.
However, not all of them are equally relevant for current PGRE subject matter.
The earliest one can be used as a great ``warmup'' test, but many test takers (myself included) don't feel that it is an accurate representation of the current test.
I personally used it at the beginning of my two-month study period but eventually dropped it from my routine by the last two weeks.
The middle two tests are better practice for your arithmetic skills (GR9677 is a beast).
The most current practice test is obviously the best representation of the current test.
GR0177 is not nearly as intense as GR9677 but still requires a large breadth of physics knowledge and a decent amount of arithmetic skill.
I saved practicing this one for my last three weeks.
By the end of week six I could do all 399 problems on the four tests and score a 990 during my practice runs.
However, my actual score wasn't nearly as impressive.

The review covers key material in classical mechanics, electricity and magnetism, optics and wave phenomena, quantum mechanics, thermodynamics, statistical mechanics, modern physics (including special relativity, atomic physics, etc\ldots), and some useful mathematical information.
This review is not limited to simply what is found in the practice exams.
It contains additional information intended to prepare the reader for exam questions that \emph{could} be asked.
Work as many problems on these subjects as possible and understand every question in the PGRE practice tests.

I tried to keep a consistent notation throughout the whole document, but when covering most of undergraduate physics I ran into several conflicting conventions in notation (i.e. \(P\) for pressure, power, and momentum).
I hope this doesn't cause confusion, but I wanted to stick to how things are commonly referred to and I feel that their meaning is obvious in context.
